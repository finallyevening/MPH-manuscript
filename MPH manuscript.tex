\documentclass[journal=jacsat,manuscript=article]{achemso}


\usepackage{biblatex}
\addbibresource{references.bib} % BibLaTeX bibliography file
\newcommand*\mycommand[1]{\texttt{\emph{#1}}}



\author{Viola Nguyen}
\email{viola.nguyen@mssm.edu}
\author{Dhoone Menezes de Sousa}
\author{Blessing Akitunde}
\author{Shyamalee Dassanayake}
\affiliation{Icahn School of Medicine}


\title{The use of Solid Phase Extraction (SPE) for detection Methylphenidate and Ritalinic Acid in small volume plasma samples }

\abbreviations{IR,NMR,UV}
\keywords{American Chemical Society, \LaTeX}


\begin{document}
\begin{abstract}
This methology describes an extraction of Methylphenidate (MPH) and Ritalinic Acid (RA), from plasma using solid-phase extraction (SPE), followed by syliation reaction. In addition an ion chromatographic method was developed for the specific GC determination of MPH and RA.  Treated plasma samples were passed through SPE cartridge with Hydrophilic-Lipophilic-Balanced (HLB) sorbent to retain and elute target analytes. Using N-Methyl-N-(trimethylsilyl)trifluoroacetamide (MSTFA) and N-Methyl-bis(trifluoroacetamide) (MBTFA) reagents, eluent was derivatized and the non-polar product was further analyzed using GC-MS. A calibration curve for MPH and RA was constructed in the range  \SIrange[range-units = brackets]{2}{250}{\micro \gram/mL}. The SPE resulted in higher extraction recovery (mean \hl  {x \%}) with \hl {\% R.S.D.s} similar in both matrix and solvent (\hl {x\%}, respectively).

\end{abstract}
\section{Introduction}
Methylphenidate, commonly known by the trade name Ritalin\textsuperscript{\textregistered}, is a central nervous system (CNS) stimulant used to treat attention deficit hyperactivity (ADHD) and narcolepsy.  MPH has four stereoisomers due to the presence of the two chiral centers: d-(R,S)-erthyro-, l-(S,R)-erythro-, d-(R,R)-threo-, and l-(S,S)-threo-MPH (HAS REFERENCE).  After administration, methylphenidate is rapidly hydrolyzed in the liver to form ritalinic acid (RA), which is then excreted out of the body via urine.  Once in the body, the half-life of MPH ranges from 1-4 hours (HAS REFERENCE, depending on the distribution). Due to the rapid degradation of MPH to RA, there is a need for an analytical method that could simultaneously detect both chemicals in biological matrices. 

Being polar compounds, MPH and RA are often analyzed using liquid chromatography mass spectrometry (LC-MS) due to the shorter preparation time and no need for derivatization. Unfortunately, not everyone would have access to an LC-MS. 

\subsection{Method and materials}
Native standards for Ritalinic Acid hydrochloride (1.0 mg/mL in MeOH) and  Methylphenidate hydrochloride (1.0 mg/mL in MeOH) were purchased from Sigma Aldrich (St. Louis MO, USA). Both labeled standards for -threo-Methylphenidate-$D_4$ HCl (\SI{100}{\micro \gram} in MeOH) and -threo-Ritalinic acid-$D_{10}$ HCl (\SI{100}{\micro \gram} in MeOH) were obtained from Cerilliant (Round Rock, TX).
BioChemed Services (Winchester, VA, USA) provided with bovine plasma. HPLC grade formic acid, ammonium hydroxide,methanol and distilled water were purchased from Fisher Chemical (Nazareth, PA). Alongside with solvents, derivatizer MBTFA [N-methyl-bis(trifluoroacetamide)] was also purchased through Fisher Chemical.  MSTFA (N-methyl-n-trimethylsilyl-trifluoroacetamide) derivatizer was purchased from RESTEK (Bellefonte, PA). Solid-phase extraction columns (Oasis PRiME HLB 3 cc Vac Cartridge, 60 mg) were acquired from Waters Corp (Milford, MA).





\subsection{Sample Preparation}


\subsection{Instrumentation}




\subsection{Instrumentation}

The detection of analytes was performed by GC–MS/MS with \hl {XXX} . Mass Hunter QQQ  software was used for the data acquisition and quantification \cite{del2007liquid}



\section{Results and discussion}
\subsection{Linearity}
\subsection{Precision and accuracy}
\subsection{Recoveries}

\section{Discussion and conclusion}
\subsection{Reliability}
\subsection{Practicality}



\section{References}
\printbibliography


\end{document}
