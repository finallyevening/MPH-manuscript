\documentclass[journal=jacsat,manuscript=article]{achemso}
\newcommand*\mycommand[1]{\texttt{\emph{#1}}}

\usepackage{siunitx}
\usepackage{color,soul}
\usepackage{lipsum}
\usepackage{titlesec}

\author{Viola Nguyen}
\email{viola.nguyen@mssm.edu}
\author{Dhoone Menezes de Sousa}
\email{dhoone.sousa@mssm.edu}
\author{Blessing Akitunde}
\email{blessing.akintunde@mssm.edu}
\affiliation{Icahn School of Medicine}

\title{The use of Solid Phase Extraction (SPE) for detection Methylphenidate and Ritalinic Acid in small volume plasma samples }

\abbreviations{IR,NMR,UV}
\keywords{American Chemical Society, \LaTeX}


\begin{document}
\begin{abstract}
This methology describes an extraction of Methylphenidate (MPH) and Ritalinic Acid (RA), from plasma using solid-phase extraction (SPE), followed by syliation reaction. In addition an ion chromatographic method was developed for the specific GC determination of MPH and RA.  Treated plasma samples were passed through SPE cartridge with Hydrophilic-Lipophilic-Balanced (HLB) sorbent to retain and elute target analytes. Using N-Methyl-N-(trimethylsilyl)trifluoroacetamide (MSTFA) and N-Methyl-bis(trifluoroacetamide) (MBTFA) reagents, eluent was derivatized and the non-polar product was further analyzed using GC-MS. A calibration curve for MPH and RA was constructed in the range  \SIrange[range-units = brackets]{2}{250}{\micro \gram/mL}. The SPE resulted in higher extraction recovery (mean \hl  {x \%}) with \hl {\% R.S.D.s} similar in both matrix and solvent (\hl {x\%}, respectively).

\end{abstract}

\section{Introduction}

\section{Experimental}
The usual experimental details should appear here.  This could
include a table, which can be referenced as Table~\ref{tbl:example}.
Notice that the caption is positioned at the top of the table.
\begin{table}
  \caption{An example table}
  \label{tbl:example}
  \begin{tabular}{ll}
    \hline
    Header one  & Header two  \\
    \hline
    Entry one   & Entry two   \\
    Entry three & Entry four  \\
    Entry five  & Entry five  \\
    Entry seven & Entry eight \\
    \hline
  \end{tabular}
\end{table}

\subsection{Method and materials}

Reference 



\subsection{Sample Preparation}
\subsubsection{GC/MS}


\subsection{Instrumentation}
\subsubsection{GC/MS}
GC

\section{Results and discussion}
\subsection{Linearity}
\subsection{Percision and accuracy}
\subsection{Recoveries}

\section{Discussion and conclusion}
\subsection{Reliability}
\subsection{Practicality}
\subsection{Cost}




\section{References}




\end{document}
